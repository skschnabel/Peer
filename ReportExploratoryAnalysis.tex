\documentclass[12pt, a4paper]{article}
\usepackage{color}

%opening
\title{Summary exploratory analysis Pear Data}
\author{Sabine K. Schnabel, Biometris}
\date{\today}

\begin{document}

\maketitle

\section*{General information about the data sets}

\subsection*{Meta data}
\begin{itemize}
\item It is not yet for sure in how far we will use the meta data file at all.
\item Excel table with information about the different orchards
\item There are 18 orchards with partial information in the data set.
\item The variables (with different degrees of information and variation): \texttt{boomvorm, boomleeftijd, onderstam, 
	grondsoort, fractie afslibbaar, fertigatie, voeding N, voeding P, voeding K, wortelsnoei, behang (\%), 
	commerciele pluk, pluk proef, CO2 prod.(ml/kg.uur), ethyleenproductie (μl/kg*h), \%droge stof, vrucht-Ca, vrucht-K, vrucht-K/Ca, vrucht-P, vrucht-Mg, vrucht-N}.
\end{itemize}

\subsection*{Large data set}
\begin{itemize}
\item Excel table with information on all measurement and indicators taken on the pears
\item The table has the following information on fruit level:
	\begin{itemize}
		\item Basic information about the record: \newline 
			\texttt{regel, vrucht nr, herkomst, product, batch, duplo, 	meetdatum, SmartFresh, \newline destructief/non-destructief, meting}
		\item General measurement (repeated at different time points): \newline 
			\texttt{kleurindex (1=groen-5=geel), \textcolor{red}{penetrometerwaarde/firmness}}
		\item Measurements from the Pigment Analyzer: \newline
			\texttt{NDVI -bloskant, NAI-bloskant, NDVI - groene zijde, NAI-groene zijde, NDVI-gemiddeld, NAI-gemiddeld}
		\item Measurements from the DA-meter (non destructive chlorophyll-analyzer): \newline
			\texttt{DA-index bloskant/red, DA-index groene zijde/green, DA-index gemiddeld, DA-indexblos-groen}
		\item Measurements from the Aweta AFS Acoustic firmness sensor: \newline
			\texttt{Stevigheid AFS-bloskant, M (g) bloskant, F0 bloskant, Stevigheid AFS-groene zijde, M (g) groene zijde, F0 groene zijde, Stevigheid AFS-gemiddeld, M (g) AFS \% verschil, F0 AFS-gemiddeld}
		\item Underwater weighing to determine density: 
			\texttt{vruchtgewicht(g), dichtheid(g/ml)}
		\item \texttt{\textcolor{red}{\% droge stof, TSS (°Brix), zetmeelindex (1-10)}}
		\item \texttt{\textcolor{cyan}{Schilvlekjes-score, Schilvlekjes index}}
		\item China simulatie:
			\texttt{rot, zwarte stelen index, hol\&bruin-score}	
	\end{itemize} 
\item Additional there is substantial information about LBA and Hue strata as well as a large amount of metabolomics 
	measurements. The latter are at batch level while the former are on fruit level.
\item A lot of the measurements are non-destructive (that incluedes the LBA/Hue measurements), but the most meaningful
	measure \texttt{firmness} is destructive and also metabolomics and other measurements (\texttt{dry matter} etc.) result 
	in the loss of the fruit.
\end{itemize}

\section*{Descriptive analysis of the data}

\subsection*{Meta data}
Since this data set is rather small and most information are only partial or do not contain enough variation, no 
further descriptive analysis was done.

\subsection*{Large data set}
These data include information on Conference pears as well as Elstar apples. For the sake of this analysis I excluded 
the apples from the data set. \newline
The resulting data set has 8720 records and includes numerous columns (with basic information and phenoytpic 
traits). To my understanding there are 4320 different pears in the data set. Since the most important measurement is 
destructive at every measurement for firmness the measured fruits are sacrificed. The fruits are uniquely 
identified through the fruit number, the orchard indication, the batch and duplo.  

\end{document}
